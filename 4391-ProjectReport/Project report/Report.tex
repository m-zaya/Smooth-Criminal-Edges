% CVPR 2022 Paper Template
% based on the CVPR template provided by Ming-Ming Cheng (https://github.com/MCG-NKU/CVPR_Template)
% modified and extended by Stefan Roth (stefan.roth@NOSPAMtu-darmstadt.de)

\documentclass[10pt,twocolumn,letterpaper]{article}
\newcommand\tab[1][0.5cm]{\hspace*{#1}}


%%%%%%%%% PAPER TYPE  - PLEASE UPDATE FOR FINAL VERSION
% \usepackage[review]{cvpr}      % To produce the REVIEW version
%\usepackage{cvpr}              % To produce the CAMERA-READY version
\usepackage[pagenumbers]{cvpr} % To force page numbers, e.g. for an arXiv version

% Include other packages here, before hyperref.
\usepackage{graphicx}
\usepackage{amsmath}
\usepackage{amssymb}
\usepackage{booktabs}


% It is strongly recommended to use hyperref, especially for the review version.
% hyperref with option pagebackref eases the reviewers' job.
% Please disable hyperref *only* if you encounter grave issues, e.g. with the
% file validation for the camera-ready version.
%
% If you comment hyperref and then uncomment it, you should delete
% ReviewTempalte.aux before re-running LaTeX.
% (Or just hit 'q' on the first LaTeX run, let it finish, and you
%  should be clear).

\usepackage[breaklinks,colorlinks]{hyperref}


% Support for easy cross-referencing
\usepackage[capitalize]{cleveref}
\crefname{section}{Sec.}{Secs.}
\Crefname{section}{Section}{Sections}
\Crefname{table}{Table}{Tables}
\crefname{table}{Tab.}{Tabs.}


%%%%%%%%% PAPER ID  - PLEASE UPDATE
\def\cvprPaperID{*****} % *** Enter the CVPR Paper ID here
\def\confName{CVPR}
\def\confYear{2022}


\begin{document}

%%%%%%%%% TITLE - PLEASE UPDATE
    \title{Smooth Criminal Edges: Developing a New Edge-preserving Image Smoothing Algorithm for Impulse Noise}

    \author{Sherry Cherniavsky\\
    University of Texas at Dallas\\
    800 W. Campbell Road\\
    Richardson, Texas 75080-3021\\
    {\tt\small Sherry.Cherniavsky@UTDallas.edu}
% For a paper whose authors are all at the same institution,
% omit the following lines up until the closing ``}''.
% Additional authors and addresses can be added with ``\and'',
% just like the second author.
% To save space, use either the email address or home page, not both
    \and
    Pax Gole\\
    {\tt\small Pax.Gole@UTDallas.edu}
    \and
    Jonah Markham\\
    {\tt\small Jonah.Markham@UTDallas.edu}
    }
    \maketitle

%%%%%%%%% ABSTRACT
 \begin{abstract}
    Abstract stuff
 \end{abstract}

%%%%%%%%% BODY TEXT
    \section{Introduction}
    Intro instructions: Describe the motivation of the project, i.e., why do you want to work on
    this problem. Then describe an overview of the framework/method/system.

%------------------------------------------------------------------------
    \section{Related Work}
    Related Work instructions: Discuss the related work of your project.

%-------------------------------------------------------------------------
    \section{Method}
    Method instructions: Describe your solution for the project. For example, describe each component of
    the framework in details. Try to use figures to illustrate the method instead of only using
    text. ”A picture is worth a thousand words”.

%-------------------------------------------------------------------------
    \section{Experiments}
    Experiments instructions: In this section, you can first describe the datasets and evaluation metrics.
    Then describe what experiments you have done for the project by adding experimental results
    to the report. You can use figures and plots to show these results.

    \section{Conclusion}
    Conclusion instructions: Describe the take-home messages of the project and conclude the report.

    \newpage
% Update the cvpr.cls to do the following automatically.
% For this citation style, keep multiple citations in numerical (not
% chronological) order, so prefer \cite{Alpher03,Alpher02,Authors14} to
% \cite{Alpher02,Alpher03,Authors14}.


%-------------------------------------------------------------------------

    {\small
    \bibliographystyle{ieee_fullname}
    \bibliography{egbib}
    }
    Note:\\
    \tab References will show up once we cite thru-out the doc.
    \newpage
    Submission note: \\
    \tab Please submit the following items to eLearning. \\
    \tab You can zip all the files. \\
    \tab\tab • (Required) Final report in pdf format \\
    \tab\tab • (Required) Source code of your project \\

    Instructions: (for easier reference)\\
    \newline
    Evaluation criteria:
    The grading will be based on the overall quality of the report
    in terms of writing, content and clarity. Failure can provide
    valuable insights. Make sure to discuss and analyze your results
    in detail, even if they are not in line with your initial expectations.\\

    Minimum page requirement: 5 pages.
    The report should be at least 5 pages with the CVPR format
    (excluding references, i.e., without references, the content
    should be at least 5 pages). You can go beyond 5 pages, but
    make sure it does not surpass 6 pages in length. (excluding references).

\end{document}
