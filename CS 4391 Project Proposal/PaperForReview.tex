\documentclass[10pt,twocolumn,letterpaper]{article}

%%%%%%%%% PAPER TYPE  - PLEASE UPDATE FOR FINAL VERSION
% \usepackage[review]{cvpr}      % To produce the REVIEW version
%\usepackage{cvpr}              % To produce the CAMERA-READY version
\usepackage[pagenumbers]{cvpr} % To force page numbers, e.g. for an arXiv version

% Include other packages here, before hyperref.
\usepackage{graphicx}
\usepackage{amsmath}
\usepackage{amssymb}
\usepackage{booktabs}

\usepackage[style=mla]{biblatex}

% It is strongly recommended to use hyperref, especially for the review version.
% hyperref with option pagebackref eases the reviewers' job.
% Please disable hyperref *only* if you encounter grave issues, e.g. with the
% file validation for the camera-ready version.
%
% If you comment hyperref and then uncomment it, you should delete
% ReviewTempalte.aux before re-running LaTeX.
% (Or just hit 'q' on the first LaTeX run, let it finish, and you
%  should be clear).
\usepackage[pagebackref,breaklinks,colorlinks]{hyperref}


% Support for easy cross-referencing
\usepackage[capitalize]{cleveref}
\crefname{section}{Sec.}{Secs.}
\Crefname{section}{Section}{Sections}
\Crefname{table}{Table}{Tables}
\crefname{table}{Tab.}{Tabs.}


%%%%%%%%% PAPER ID  - PLEASE UPDATE
\def\cvprPaperID{*****} % *** Enter the CVPR Paper ID here
\def\confName{CVPR}
\def\confYear{2022}


\begin{document}

%%%%%%%%% TITLE
\title{SCE: Developing a New Edge-preserving Image Smoothing Algorithm for Impulse Noise}

\author{Sherry Cherniavsky\\
University of Texas at Dallas\\
800 W. Campbell Road\\
Richardson, Texas 75080-3021\\
{\tt\small Sherry.Cherniavsky@UTDallas.edu}
% For a paper whose authors are all at the same institution,
% omit the following lines up until the closing ``}''.
% Additional authors and addresses can be added with ``\and'',
% just like the second author.
% To save space, use either the email address or home page, not both
\and
Pax Gole\\
{\tt\small Pax.Gole@UTDallas.edu}
\and
Jonah Markham\\
{\tt\small Jonah.Markham@UTDallas.edu}
}
\maketitle

%%%%%%%%% ABSTRACT
%\begin{abstract}
%   The ABSTRACT is to be in fully justified italicized text, at the top of the left-hand column, below the author and affiliation information.
%   Use the word ``Abstract'' as the title, in 12-point Times, boldface type, centered relative to the column, initially capitalized.
%   The abstract is to be in 10-point, single-spaced type.
%   Leave two blank lines after the Abstract, then begin the main text.
%   Look at previous CVPR abstracts to get a feel for style and length.
%\end{abstract}

%%%%%%%%% BODY TEXT
\section{Problem Statement}
This research investigates techniques to enhance edge-preserving image
denoising algorithms, in order to better restore images that have been
corrupted with impulse noise, while maintaining sharp edges and fine details.

%------------------------------------------------------------------------
\section{Approach}
The approach leverages the characteristics of impulse noise to enhance
edge-preserving smoothing techniques. Noisy pixels will be detected then
replaced using surrounding pixels, rather than indiscriminately filtering
all pixels. This should better preserve edges compared to basic techniques
like bilateral filtering.

%-------------------------------------------------------------------------
\section{Data}
The image dataset will consist of photographs from the USC-SIPI image
database corrupted with varying levels of salt & pepper noise.
This will allow testing under different noise conditions.

%-------------------------------------------------------------------------
\section{Evaluation}
The final results will be evaluated by benchmarking against common
filtering techniques, including staple methods such as applying Gaussian,
Bilateral, and Non-Local Means filters uniformly across all pixels.
Quantitative metrics and visual inspection will compare output images to
the originals, judging the performance of our proposed technique against
these conventional benchmarks. This comparative analysis will reveal how
well our approach preserves edges and details relative to establish filters
applied indiscriminately without excluding noisy pixels.

%-------------------------------------------------------------------------
\section{References}
Will figure this out tmrw


\end{document}
